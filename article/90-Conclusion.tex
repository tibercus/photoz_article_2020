\section{Conclusions}\label{sec:results}

Проведено исследование, построение и сравнение моделей вероятностных прогнозов фотометрических красных смещений (photo-z) на основе алгоритма случайного леса  с использованием данных современных астрономических обзоров SDSS, PanStarrs и DESI Legacy Survey для построения трехмерной карты квазаров.

Предложена модель photo-z, значительно превосходящая (в ~2 раза) по точности (метрики точечных прогнозов — нормализованное медианное абсолютное отклонение NMAD и доля выбросов n>0.15) лучшие модели (SOTA) известные в литературе. Для рентгеновских источников в тестовой области неба Stripe82X получена точность NMAD = 0.034 / 0.064 / 0.067 и n>0.15 = 0.079 / 0.170 / 0.163 для предложенной модели / шаблонной модели Ananna, 2017 / нейросетевой модели Brescia, 2019, соответственно.

%Предложен алгоритм дополнительной классификации далеких объектов (красное смещение больше 3), превосходящий классификацию по мере достовероности прогноза zConf (предложенный алгоритм: ROC AUC = 0.98; классификация по zConf: ROC AUC = 0.94). При полноте 0.93 достигнуто получено уменьшение доли звезд, ложно классифицированных как квазары (предложенный алгоритм: FPR = 0.10; классификация по zConf: FPR = 0.21).

%Предложен алгоритм постобработки вероятностных прогнозов для улучшения калибровки Temperature Scaling, показана его работоспособность.