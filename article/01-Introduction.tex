\section{Introduction}\label{sec:intro}

В 2019 году была запущена российско-германская орбитальная обсерватория "Спектр-Рентген-Гамма", одной из научных задач которой является составление трехмерной карты активных ядер галактик (квазаров) \achtung{ссылка на сайт SRG}. Карта квазаров необходима для изучения крупномасштабной структуры и эволюции Вселенной. Кроме этого перед учеными стоит задача детального изучения наиболее далеких квазаров -- изучение только одного такого объекта уже является весомым научным достижением. Для выполнения этих задач необходимо измерять космологическое красное смещение квазаров.

Красное смещение может быть напрямую измерено из спектральных данных (спектральное красное смещение, spec-z) или оценено по фотометрическим данным, т.е. по изображениям звездного неба, полученным в широких диапазонах излучения (оптика, рентген, инфракрасный диапазон). Такие оценки называются фотометрическим красным смещением (photo-z). Метод фотометрии позволяет получать данные быстрее и для слабых источников, таким образом, основным преимуществом photo-z является то, что оно может быть получено для большего числа объектов, но при этом является менее точным, чем более ресурсозвтратные spec-z \cite{bib:nature_photoz}.

В задаче измерения photo-z возникает проблема мультимодальности прогнозов. На рисунке \ref{fig:sed_redshift_degeneracy} представлен пример спектров двух галактик \achtung{mesch - Нужен пример для 2-х рентгеновских квазаров}, находящихся на разном красном смещении. Видно, что в некотором диапазоне излучения их спектры сильно похожи, и, как следствие, сильно похожи фотометрические признаки (отмечены черными точками на графике). Таким образом получается неоднозначное соответствие прогнозов признакам, и точечная оценка, построенная на основе этих признаков, скорее всего, будет неверной. Поэтому необходимо прогнозировать не само значение красного смещения объекта, а распределение значения красного смещения объекта. Такой прогноз называется вероятностным фотометрическим красным смещением (вероятностный photo-z). Вероятностный photo-z позволит вычислять доверительные интервалы, оценивать надежность прогнозов.

\achtung{Увидел замечание про то, что архитектура SRGZ будет в другой статье. Вообще описывать не надо?}

\achtung{Дописать план рааботы}