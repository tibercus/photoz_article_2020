\section{Related work}\label{sec:related_work}

\achtung{Пока так, потом соединим с Introduction}

Общие подходы к решению задачи прогноза photo-z описываются в \cite{bib:nature_photoz}. Все решения, предлагаемые для решения данной задачи можно разделить на 2 группы: основанные на использовании физических моделей (Physically motivated methods, шаблонные модели) и основанные на использовании данных (Data driven methods), т.е. с применением технологий машинного обучения. На рисунке \ref{fig:photo-z_in_a_nutshell} представлена общая схема методов photo-z. Сначала спектральная обучающая выборке ассоциируются с фотометрией для получения размеченной выборки, где в качестве признаков выступают фотометрические данные, а в качестве истинных значений целевого признака -- spec-z (Spectroscopic training sample + associated photometry). Далее на основе этой выборки и выбранного метода photo-z, который может являться шаблонным методом (Templates + transmission), методом на основе машинного обучения (Machine Learning) или гибридным методом (Hybrid algorithms), строится модель photo-z, отображающая фотометрические признаки (цвета) в photo-z (colors-redshift mapping model). Затем этой модели подаются на вход фотометрические признаки объектов целевой выборки (Input photometry) для получения точечного и вероятностного прогнозов photo-z (\(z\), \(PDF_z\)).

Суть методов на основе шаблонов состоит в использовании спектров типовых объектов различных классов. На основе спектра строится шаблон -- физическая модель предсказывающая значение фотометрических признаков по значению красного смещения, составляется библиотека шаблонов. Далее производится поиск оптимального шаблона, на котором достигается минимальная ошибка между предсказанными фотометрическими признаками и измеренными, путем минимизации критерия \(\mathcal{X}^2\).

Множество методов photo-z на основе алгоритмов машинного обучения состоит в основном из алгоритмов, основанных на ансамблях деревьев решений (например, случайный леc, деревья бустинга), и нейросетевых алгоритмов. Нейронные сети могут применяться как к табличным признакам, так и к сырым фотометрическим данным (изображения). Выбор таких алгоритмов многими исследователями основан на том, что эти алгоритмы давали хорошие результаты в прошлых работах.

В работе \cite{bib:nature_photoz} было проведено сравнение разных методов, по большей части шаблонных, на данных разных астрономических каталогов. Результаты сравнения представлены на рис. \ref{fig:salvato_comparison}. Можно видеть, что качество моделей сильно зависит от данных, на которых модель была построена. Поэтому в данной задаче очень важен выбор данных для построения моделей.

На данный момент две лучшие по точности модели photo-z для рентгеновских объектов, известные в литературе, (SOTA) - модель Ananna, 2017 \cite{bib:ananna2017}, которая основана на шаблонном методе photo-z из пакета LePhare, и модель Brescia, 2019 \cite{bib:brescia2018}, которая основана на использовании многослойного персептрона из пакета METAPhoR, описание которого приводится ниже. Данными моделями получены оценки photo-z для рентгеновских объектов поля Stripe82X, описанной в \cite{bib:ananna2017}, и являющейся основной тестовой выборкой рентгеновских объектов, поэтому при проведении исследования необходимо ориентироваться на результаты этих двух работ.

В статье LSST -- одного из крупнейших фотометрических обзоров неба в оптическом диапазоне излучения следующего поколения, \cite{bib:lsst} приводится описание и сравнение различных методов вероятностных photo-z. Ниже представлен обзор и сравнение этих методов.

В пакете ANNz2 \cite{bib:annz2} реализованы методы на основе многослойного персептрона, деревьев бустинга (boosted decision trees) и регрессии k-ближайших соседей. Реализованные в пакете алгоритмы позволяют оценивать достоверность прогноза, и в качестве финального прогноза дается лучший из прогнозов ансамбля моделей или взвешенная сумма прогнозов. В сравнении использовался ансамбль из пяти многослойных персептронов с архитектурой слоёв 6:12:12:1, где на вход подавались данные шести фильтров ugrizy и на выходном слое получается прогноз photo-z. Каждый персептрон обучался со случайной инициализацией параметров.

В пакете CMNN \cite{bib:cmnn} для прогноза photo-z используется метод ближайших соседей. В качестве признаков выступают, так называемые цвета (отношение яркостей между разными фильтрами), расстоянием между соседями выступает расстояние Махалонобиса
\begin{equation}
    D_M = \sum_j^{N_{colors}} \frac{(c_{j,train} - c_{j,test})^2}{(\delta c_{j,test})^2},
\end{equation}
и в качестве прогноза дается взвешенная сумма целевых признаков соседей с весами, обратными расстояниям до них.
%\achtung{пока не очень въехал в описание. Дописать}

FlexZBoost \cite{bib:flexzboost} представляет собой набор общих методов адаптации алгоритмов оценки условного среднего для оценки условной плотности распределения целевого признака. Для этого неизвестная функция плотности раскладывается по ортонормированному базису. В статье было использовано преобразование Фурье. Далее коэффициенты Фурье вычисляются как точечная оценка с использованием выбранного алгоритма машинного обучения (в статье использовался xgboost).

Алгоритм GPz \cite{bib:gpz} основан на использовании гауссовых процессов. 

METAPhoR \cite{bib:metaphor} представляет собой алгоритм вероятностных прогнозов photo-z, позволяющий адаптировать алгоритмы регрессии для получения вероятностных оценок. По умолчанию используется многослойный персептрон обучение которого происходит квазиньютоновским алгоритмом (Multi Layer Perceptron with Quasi Newton Algorithm, MLPQNA) минимизацией ошибки MSE с применением L2-регуляризации. Для получения вероятностного прогноза признаки объекта разыгрываются в соответствии с ошибками измерений на них.

В пакете SkyNet \cite{bib:skynet} используются многослойный персептрон, обучаемый методом градиентного спуска второго порядка минимизацией кросс-энтропийной функции потерь, то есть решается задача классификации. Область определения целевого признака разбивается на 200 интервалов, объект принадлежит к некоторому классу, если его целевой признак принадлежит к соответствующему интервалу. Вероятностный прогноз получается из двухсот выходов выходного слоя с функцией активации softmax нейронной сети, на вход алгоритма подаются измерения и ошибки измерения шести фотометрических признаков (всего 12 входов).

В TPZ \cite{bib:tpz} используется алгоритм случайного леса без ограничения глубины. Вероятностный прогноз получается путем решения множества задач бинарной классификации: область определения целевого признака разбивается на интервалы, и для каждого интервала строится модель классификации, которая определяет, принадлежит ли значение целевого признака заданному интервалу или не принадлежит. Вероятностный прогноз представляет собой набор вероятностей принадлежности значения целевого признака тому или иному интервалу.

\achtung{Дописать статью с Машечкиным.}

\achtung{Может, весь Related Works подсократить?}