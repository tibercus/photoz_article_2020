\usepackage{fontspec} % XeTeX
\usepackage{xunicode} % Unicode для XeTeX
\usepackage{xltxtra}  % Верхние и нижние индексы
\usepackage{pdfpages} % Вставка PDF
\usepackage{longtable}
\usepackage{rotating}
\usepackage{subcaption} % for subfigure

\usepackage{listings} % Оформление исходного кода
\lstset{
    basicstyle=\small\ttfamily, % Размер и тип шрифта
    breaklines=true,            % Перенос строк
    tabsize=2,                  % Размер табуляции
    frame=single,               % Рамка
    literate={--}{{-{}-}}2,     % Корректно отображать двойной дефис
    literate={---}{{-{}-{}-}}3  % Корректно отображать тройной дефис
}

% Русский язык
\usepackage{polyglossia}
\setdefaultlanguage{russian}

% Шрифты, xelatex
\defaultfontfeatures{Ligatures=TeX}
\setmainfont{Times New Roman}
\newfontfamily\cyrillicfont{Times New Roman}
% \setsansfont{Liberation Sans}
\setmonofont{FreeMono}

\usepackage{geometry}
\geometry{left=3cm}
\geometry{right=1.5cm}
\geometry{top=2cm}
\geometry{bottom=2cm}

\renewcommand{\baselinestretch}{1.3} % Полуторный межстрочный интервал
\parindent 1.27cm % Абзацный отступ

\usepackage{indentfirst} % включает отступ после заголовка
\usepackage{hyperref}
\urlstyle{same}

\sloppy             % Избавляемся от переполнений
\hyphenpenalty=1000 % Частота переносов
\clubpenalty=10000  % Запрещаем разрыв страницы после первой строки абзаца
\widowpenalty=10000 % Запрещаем разрыв страницы после последней строки абзаца

% Пути к каталогам с изображениями
\usepackage{graphicx} % Вставка картинок и дополнений
\graphicspath{{resources/}}
\usepackage{wrapfig}

% Заголовки секций в оглавлении в верхнем регистре
\usepackage{textcase}
\makeatletter
\let\oldcontentsline\contentsline
\def\contentsline#1#2{
    \expandafter\ifx\csname l@#1\endcsname\l@section
        \expandafter\@firstoftwo
    \else
        \expandafter\@secondoftwo
    \fi
    {\oldcontentsline{#1}{\MakeTextUppercase{#2}}}
    {\oldcontentsline{#1}{#2}}
}
\makeatother

\usepackage{titlesec}
\titleformat{\section}{\centering\uppercase}{\thesection.\quad}{0pt}{}
\titlespacing\section{1.27cm}{0cm}{0cm}
\titleformat{\subsection}{}{\thesubsection.\quad}{0pt}{}
\titlespacing\subsection{1.27cm}{0cm}{0cm}
\titleformat{\subsubsection}{}{\thesubsubsection.\quad}{0pt}{}
\titlespacing\subsubsection{1.27cm}{0cm}{0cm}

\newcommand{\anonsection}[1]{
    \phantomsection
    \section*{#1}
    \addcontentsline{toc}{section}{#1}
}

\pagenumbering{arabic}

% Содержание
\usepackage{tocloft}
% СОДЕРЖАНИЕ
\renewcommand{\cfttoctitlefont}{\hspace{0.42\textwidth}\MakeTextUppercase}
% Имена секций в содержании не жирным шрифтом
\renewcommand{\cftsecfont}{\hspace{0pt}}
% Точки для секций в содержании
\renewcommand\cftsecleader{\cftdotfill{\cftdotsep}} 
% Номера страниц не жирные
\renewcommand\cftsecpagefont{\mdseries}
\renewcommand{\contentsname}{Содержание}
% Заголовок содержания

\setcounter{tocdepth}{3}
\setcounter{page}{1} % нумерация со второй страницы

\usepackage{float}
\usepackage{amssymb} % for mathbb font

\usepackage{amsmath}
\DeclareMathOperator*{\argmax}{argmax} % for argmax

% Библиография: отступы и межстрочный интервал
\makeatletter
\renewenvironment{thebibliography}[1]
    {\section*{\refname}
        \list{\@biblabel{\@arabic\c@enumiv}}
           {\settowidth\labelwidth{\@biblabel{#1}}
            \leftmargin\labelsep
            \itemindent 16.7mm
            \@openbib@code
            \usecounter{enumiv}
            \let\p@enumiv\@empty
            \renewcommand\theenumiv{\@arabic\c@enumiv}
        }
        \setlength{\itemsep}{0pt}
    }
\makeatother

% Оформление библиографии и подрисуночных записей через точку
\makeatletter
\renewcommand*{\@biblabel}[1]{\hfill#1.}
\renewcommand*\l@section{\@dottedtocline{1}{1em}{1em}}
%\renewcommand{\thefigure}{\thesection.\arabic{figure}} % Формат рисунка секция.номер
\renewcommand{\thefigure}{\arabic{figure}}
%\renewcommand{\thetable}{\thesection.\arabic{table}}   % Формат таблицы секция.номер
\renewcommand{\thetable}{\arabic{table}}
\def\redeflsection{\def\l@section{\@dottedtocline{1}{0em}{10em}}}
\makeatother

% Списки
\usepackage{enumitem}
\setlist[enumerate,itemize]{leftmargin=12.7mm} % Отступы в списках

\makeatletter
    \AddEnumerateCounter{\asbuk}{\@asbuk}{м)}
\makeatother
\setlist{nolistsep}                           % Нет отступов между пунктами списка
\renewcommand{\labelitemi}{--}                % Маркер списка --
\renewcommand{\labelenumi}{\asbuk{enumi})}    % Список второго уровня
\renewcommand{\labelenumii}{\arabic{enumii})} % Список третьего уровня

% Содержание
\usepackage{tocloft}
\renewcommand{\cfttoctitlefont}{\hspace{0.38\textwidth}\MakeTextUppercase} % СОДЕРЖАНИЕ
\renewcommand{\cftsecfont}{\hspace{0pt}}            % Имена секций в содержании не жирным шрифтом
\renewcommand\cftsecleader{\cftdotfill{\cftdotsep}} % Точки для секций в содержании
\renewcommand\cftsecpagefont{\mdseries}             % Номера страниц не жирные
\setlength{\cftaftertoctitleskip}{0em}
\setcounter{tocdepth}{3}                            % Глубина оглавления, до subsubsection

% Список табличного материала
\renewcommand{\cftlottitlefont}{\hspace{0.2\textwidth}\MakeTextUppercase}
\renewcommand{\cfttabfont}{Таблица }

% Формат подрисуночных надписей
\RequirePackage{caption}
\DeclareCaptionLabelSeparator{defffis}{ -- } % Разделитель
\captionsetup[figure]{justification=centering, labelsep=defffis, format=plain} % Подпись рисунка по центру
\captionsetup[table]{justification=raggedright, labelsep=defffis, format=plain, singlelinecheck=false} % Подпись таблицы слева
\addto\captionsrussian{\renewcommand{\figurename}{Рисунок}} % Имя фигуры

% Пользовательские функции
\newcommand{\addimg}[4]{ % Добавление одного рисунка
    \begin{figure}
        \centering
        \includegraphics[width=#2\linewidth]{#1}
        \caption{#3} \label{#4}
    \end{figure}
}
\newcommand{\addimghere}[4]{ % Добавить рисунок непосредственно в это место
    \begin{figure}[H]
        \centering
        \includegraphics[width=#2\linewidth]{#1}
        \caption{#3} \label{#4}
    \end{figure}
}
\newcommand{\addtwoimghere}[5]{ % Вставка двух рисунков
    \begin{figure}[H]
        \centering
        \includegraphics[width=#3\linewidth]{#1}
        \hfill
        \includegraphics[width=#3\linewidth]{#2}
        \caption{#4} \label{#5}
    \end{figure}
}